\ \\
\ \\
\label{pagresum}
\noindent{\LARGE \sc Resumen}\\
\ \\
\ \\
Denominamos ``legacy Software'', o ``Software Heredado'', a programas  vigentes tras veinte, treinta y hasta cuarenta a�os. Generalmente son programas grandes, complejos, que desempe�an una tarea crucial en la organizaci�n a la que pertenecen. Particularmente en las �reas cient�ficas, con problemas de c�lculo intensivo, una vez que un programa arroja resultados correctos, no suelen existir modificaciones al c�digo. Uno de los lenguajes de programaci�n mas ampliamente adoptados por la comunidad cient�fica es Fortran y justamente en este lenguaje est�n programados gran cantidad de los problemas de c�mputo intensivo que han devenido en sistemas legacy. 

Las aplicaciones legacy, o heredadas, debido al paso de una cierta cantidad de tiempo, enfrentan finalmente la problem�tica de dar respuesta a cambios ambientales, donde deben ser modernizadas o considerar terminar su ciclo de vida. El trabajo de modernizar un sistema legacy puede tener una envergadura variable, dependiendo de la complejidad del sistema y del nuevo ambiente donde vaya a funcionar. La modernizaci�n de una aplicaci�n legacy puede verse como un proceso de optimizaci�n de la aplicaci�n, s�lo que para una plataforma diferente de aquella para la cual fue construida.

Esta tesis presenta la modernizaci�n de una aplicaci�n cient�fica del campo de la Din�mica de Fluidos, la cual se encuentra dentro de las aplicaciones de Computo de Altas Prestaciones (HPC en ingl�s), desarrollada entre mediados y fines de la d�cada de 1990, escrita en lenguaje Fortran para una plataforma de computaci�n con recursos limitados. Se describe el proceso de optimizaci�n de la aplicaci�n, pasando en primer instancia por una optimizaci�n serial y luego por una optimizaci�n paralela, de manera que pueda aprovechar recursos que no estaban contemplados en su dise�o original. La implementaci�n de multiprocesamiento se realiza con la interfaz de programaci�n paralela OpenMP. Se modifica el c�digo lo menos posible para asegurar que el autor y usuario de la aplicaci�n pueda seguir manteni�ndola.

El proceso de optimizaci�n e implementaci�n de multiprocesamiento se ilustra con pruebas de ejecuci�n de las distintas versiones de la aplicaci�n: versi�n original, versi�n optimizada serialmente y versi�n optimizada paralelamente, observando la mejora en los tiempos de ejecuci�n y el impacto en la utilizaci�n de los recursos computacionales con distintos tama�os de problema.

\vfill
\pagebreak
