\chapter{Referencia del Lenguaje Fortran}\label{apen1}

En este anexo se presenta una referencia resumida del lenguaje Fortran en su est�ndar 77, el cual es utilizado en la aplicaci�n objeto de estudio de este trabajo de tesis. Particularmente se detallan las sentencias mas utilizadas en la aplicaci�n en cuesti�n. Para mayor referencia se puede ampliar consultando \citep{Ques} o la definici�n del est�ndar, que puede consultarse en \url{https://www.fortran.com/F77_std/f77_std.html}.

\section{Estructuas de Especificaci�n}
\subsection{COMMON}
Define una o m�s �reas contiguas de memoria, o bloques. Tambi�n define el orden en el que las variables, arrays y records aparecen en un bloque com�n.
Dentro de un programa, puede haber un bloque COMMON sin nombre, pero si existen m�s, se les ha de asignar un nombre. Esta instrucci�n, seguida por una serie de instrucciones de especificaci�n, asigna valores iniciales a entidades de bloques comunes con nombre y a la vez, establece y define estos bloques.\\

La sintaxis es:
\begin{lstlisting}[style=For, numbers=none]
COMMON [/nomb/] list [[,]/[nomb1]/list1] . . .
\end{lstlisting}
\ \\
donde:\\
\textbf{nomb} es un nombre simb�lico.\\
\textbf{list} es una lista de nombres de variables, nombres de arrays y declaradores de array.\\
Cuando se declaran bloques comunes con el mismo nombre en diferentes unidades de programa, estos comparten la misma �rea de memoria cuando se combinan en un programa ejecutable.

\section{Estructuras de Control}
\subsection{DO indexado}
Controla el procesamiento iterativo, o sea, las instrucciones de su rango se ejecutan un n�mero especificado de veces. Tiene la forma:
\begin{lstlisting}[style=For, numbers=none]
DO [s[,]] v = e1 , e2 [,e3 ]
\end{lstlisting}
\ \\
donde:\\
\textbf{s} es la etiqueta de una instrucci�n ejecutable, que ha de estar en la misma unidad de
programa.\\
\textbf{v} es una variable entera o real, que controla el bucle (�ndice).\\
\textbf{e1 ,e2 ,e3} son expresiones aritm�ticas.\\
La variable \textbf{v} es la variable de control, \textbf{e1} es el valor inicial que toma \textbf{v}, \textbf{e2} es el valor final y \textbf{e3} es el incremento o paso, que no puede ser cero. Si se omite \textbf{e3}, su valor por defecto es 1.
El rango de una DO incluye todas las instrucciones que siguen a la misma DO hasta la instrucci�n terminal, la �ltima del rango.\\
La instrucci�n terminal no puede ser:\\
\begin{itemize}
\item una GOTO incondicional o asignada.
\item un IF aritm�tico.
\item un bloque IF.
\item ELSE , ELSE IF , END IF , RETURN , STOP, END , otra DO.
\end{itemize}
\ \\
El n�mero de ejecuciones del rango de una DO, llamado contador de iteraciones viene dado por:\\
MAX(INT($(e2 - e1 + e3 )/e3$), 0)\\
donde INT(x) representa la funcio?n parte entera de x.
Y las etapas seguidas en la ejecucio?n son las siguientes:
\begin{enumerate}[1.]
 \item Se eval�a el contador = INT($(e2 - e1 + e3)/e3$)
 \item Se hace v = e1
 \item Si contador es mayor que cero, entonces:
  \begin{enumerate}[a)]
   \item Ejecutar las instrucciones del rango del bucle
   \item Asignar v = v + e3
   \item Decrementar el contador (contador=contador-1). Si contador es mayor que cero, repetir el bucle.  
  \end{enumerate}
\end{enumerate}

\subsection{GOTO incondicional}
Las instrucciones GOTO transfieren el control dentro de una unidad de programa. Dependiendo del valor de una expresi�n, el control se transfiere, bien a la misma instrucci�n siempre, o bien a una de un determinado conjunto de instrucciones.
En el caso del GOTO incondicional, transfiere el control a la misma instrucci�n cada vez que se ejecuta. Tiene la forma:
\begin{lstlisting}[style=For, numbers=none]
GOTO s
\end{lstlisting}
\ \\
donde \textbf{s} es la etiqueta de una instrucci�n ejecutable que est� en la misma unidad de programa de la instrucci�n GOTO.


\section{Entrada Salida y Manejo de Archivos}