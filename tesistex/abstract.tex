\ \\
\ \\
\label{pagsumm}
\noindent{\LARGE \sc Abstract}\\
\ \\
%Description of the dissertation of up to 500 words.\\
Computer programs which stay in production for a very long time, risking obsolescence in spite of the surrounding technological change, are usually called Legacy Software. Legacy applications are mission critical, and as such are hard to replace. They remain forcibly in operation, some times during decades.

They are usually big, complex programs dealing with crucial problems of the organization they belong to. Once the correct results are reached, scientific, compute-intensive programs do not suffer any new modifications. However, legacy applications will eventually face the issue of adapting to environmental changes, being modernized, or else reach their end of life. The task of modernizing a legacy application can be a variably sized one, depending on the system complexity and on the new target environment. Modernizing a legacy application can be regarded as optimization, only for a different platform than the one it was built for.

This work presents the optimization of a scientific application taken from the field of Fluid Dynamics. The application was developed in Fortran as a part of a PhD work. The application under study analyzes the behavior of an horizontal axis wind turbine, modeling the inviscid flow encircling the turbine blades by means of the panels method. By optimizing the application, we seek to enhance performance and resource utilization.

We describe how the application has underwent a serial optimization of the original Fortran code, and then a parallel optimization for shared memory machines, thus coming to utilize resources that were not contemplated in its original design. The multiprocessing implementation is done with the OpenMP parallel programming interface, which provides a portable and scalable model for the development of parallel, shared-memory applications. 

During optimization, different versions of the application have been developed: a serial-optimized version, and a parallel-optimized version. The serial version is optimized by leveraging the, now greater, amount of available RAM memory in modern shared-memory equipment. The parallel version's performance is boosted by simultaneous execution of data-independent loops on multiple cores. Performance evaluation for the parallel version shows a significant performance enhancement with regards to the serial version for small problem sizes. However, performance is lower for big problem sizes, suggesting new feasible optimizations.

\vfill
\pagebreak
