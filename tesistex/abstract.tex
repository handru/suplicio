\ \\
\ \\
\label{pagsumm}
\noindent{\LARGE \sc Abstract}\\
\ \\
\ \\

\ \\

\ \\
\ \\
%Description of the dissertation of up to 500 words.\\


Programs which are still running after several decades are usually called Legacy Software. They are usually big, complex programs performing crucial tasks in their owner organization. Once a program is found to yield correct results, its code is seldom modified, especially in scientific areas, where compute-intensive problems often belong. Fortran is one of the most widely adopted languages in the scientific community. Many compute-intensive problems having become legacy systems are programmed in this language.

Due to the passing of time, legacy applications eventually face environmental changes which they must account for, at which time they must be modernized, or their end of life has to be considered. Modernizing a legacy system entails a work of a variable size, depending on the system's complexity and the new proposed environment. Modernizing a legacy application can be considered as an optimization process, only for a different platform than that it was built for.

This thesis introduces the modernization of a scientific application taken from the field of Fluid Dynamics, a discipline commonly related to High Performance Computing. This application was developed in the late 90s, in Fortran, for a rather resource-limited computing platform. Here the optimization process is described, starting with serial optimization, and then performing parallel optimization, so that new resources, not envisioned in the original design, can be leveraged. Multiprocessing is implemented through the OpenMP parallel programming interface. The application's code has been modified as little as possible so as to ensure that the application's author and user may continue to maintain it.

The development of optimization and multiprocessing is illustrated with execution performance tests at several stages of the process: original, serially optimized and parallel optimized versions, characterizing the execution speedup obtained and the utilization of computational resources at distinct problem sizes.

\vfill
\pagebreak
